% در این فایل، دستورها و تنظیمات مورد نیاز، آورده شده است.
%-------------------------------------------------------------------------------------------------------------------
\usepackage {indentfirst}
\usepackage{multirow}
% در ورژن جدید زی‌پرشین برای تایپ متن‌های ریاضی، این سه بسته، حتماً باید فراخوانی شود
\usepackage{amsthm,amssymb,amsmath}

\usepackage{xcolor,colortbl}
\definecolor{Gray}{gray}{0.85}

% فراخوانی بسته زی‌پرشین و تعریف قلم فارسی و انگلیسی

% بسته‌ای برای تنطیم حاشیه‌های بالا، پایین، چپ و راست صفحه
\usepackage[top=35mm, bottom=37mm, left=25mm, right=30mm]{geometry}
% بسته‌‌ای برای ظاهر شدن شکل‌ها و تصاویر متن
\usepackage{graphicx}
% بسته‌ای برای رسم کادر
\usepackage{framed} 

% بسته‌‌ای برای چاپ شدن خودکار تعداد صفحات در صفحه «معرفی پایان‌نامه»

\usepackage{lastpage}
% بسته‌ و دستوراتی برای ایجاد لینک‌های رنگی با امکان جهش
%\usepackage[pagebackref=false,colorlinks,linkcolor=blue,citecolor=blue]{hyperref}
% چنانچه قصد پرینت گرفتن نوشته خود را دارید، خط بالا را غیرفعال و  از دستور زیر استفاده کنید چون در صورت استفاده از دستور زیر‌‌، 
% لینک‌ها به رنگ سیاه ظاهر خواهند شد که برای پرینت گرفتن، مناسب‌تر است
\usepackage[pagebackref=false]{hyperref}
% بسته‌ لازم برای تنظیم سربرگ‌ها
\usepackage{fancyhdr}
%
\usepackage[toc,page]{appendix} %appendix
\usepackage{setspace}
\usepackage{algorithm}
\usepackage{algorithmic}

\usepackage{subfigure}
\usepackage[subfigure,titles]{tocloft}
\cftsetindents{figure}{0em}{3.5em}
\cftsetindents{table}{0em}{3.5em}

% CONFLICT with SUBFIGURE !
%\usepackage{subcaption}
   %\captionsetup{textfont = sl} % use slanted font shape automatically for all captions
%\usepackage[noabbrev]{cleveref}

\usepackage{caption}
\captionsetup[table]{labelsep=space}
\captionsetup[figure]{labelsep=space}
\captionsetup[table]{position=top}   %% or below




% بسته‌ای برای ظاهر شدن «مراجع» و «نمایه» در فهرست مطالب
\usepackage[nottoc,notlof,notlot]{tocbibind}


% دستورات مربوط به ایجاد نمایه
\usepackage{makeidx}
\makeindex
%%%%%%%%%%%%%%%%%%%%%%%%%%
\usepackage{backnaur}

\usepackage{longtable}
\usepackage{titlesec}

%\usepackage[Kashida=on]{xepersian}
\usepackage{xepersian}

%\settextfont[Scale=1]{XB Niloofar}
\settextfont[Scale=0.9]{Persian Modern}
%\settextfont[Scale=1]{HM FElmi}
\setlatintextfont[Scale=0.9]{Latin Modern Roman}

%%%%%%%%%%%%%%%%%%%%%%%%%%
% چنانچه می‌خواهید اعداد در فرمول‌ها، انگلیسی باشد، خط زیر را غیرفعال کنید
\setdigitfont[Scale=0.857]{Persian Modern}
%%%%%%%%%%%%%%%%%%%%%%%%%%
% تعریف قلم‌های فارسی و انگلیسی اضافی برای استفاده در بعضی از قسمت‌های متن
\defpersianfont\titlefont[Scale=0.857]{B Titr}
% \defpersianfont\iranic[Scale=1.1]{XB Zar Oblique}%Italic}%
% \defpersianfont\nastaliq[Scale=1.2]{IranNastaliq}

%%%%%%%%%%%%%%%%%%%%%%%%%%
% دستوری برای حذف کلمه «چکیده»
\renewcommand{\abstractname}{}
% دستوری برای حذف کلمه «abstract»
%\renewcommand{\latinabstract}{}
% دستوری برای تغییر نام کلمه «اثبات» به «برهان»
\renewcommand\proofname{\textbf{برهان}}
% دستوری برای تغییر نام کلمه «کتاب‌نامه» به «مراجع»
\renewcommand{\bibname}{\hfil  مراجع}
%\renewcommand{\bibname}{\normalsize \begin{center} مراجع \end{center}}
% دستوری برای تعریف واژه‌نامه انگلیسی به فارسی
\newcommand\persiangloss[2]{#1\dotfill\lr{#2}\\}
% دستوری برای تعریف واژه‌نامه فارسی به انگلیسی 
\newcommand\englishgloss[2]{#2\dotfill\lr{#1}\\}
% تعریف دستور جدید «\پ» برای خلاصه‌نویسی جهت نوشتن عبارت «پروژه/پایان‌نامه/رساله»
\newcommand{\پ}{پروژه/پایان‌نامه/رساله }

\renewcommand\appendixname{پیوست}

%\newcommand\BackSlash{\char`\\}

%%%%%%%%%%%%%%%%%%%%%%%%%%
\SepMark{-}

% تعریف و نحوه ظاهر شدن عنوان قضیه‌ها، تعریف‌ها، مثال‌ها و ...
\theoremstyle{definition}
\newtheorem{definition}{تعریف}[section]
\newtheorem{theorem}[definition]{قضیه}
\newtheorem{lemma}[definition]{لم}
\newtheorem{proposition}[definition]{گزاره}
\newtheorem{corollary}[definition]{نتیجه}
\newtheorem{remark}[definition]{ملاحظه}
\theoremstyle{definition}
\newtheorem{example}[definition]{مثال}

%\renewcommand{\theequation}{\thechapter-\arabic{equation}}
%\def\bibname{مراجع}
\numberwithin{algorithm}{chapter}
\def\listalgorithmname{فهرست الگوریتم‌ها}
\def\listfigurename{فهرست اشکال}
\def\listtablename{فهرست جداول}

%%%%%%%%%%%%%%%%%%%%%%%%%%%%
% دستورهایی برای سفارشی کردن سربرگ صفحات
% \newcommand{\SetHeader}{
% \csname@twosidetrue\endcsname
% \pagestyle{fancy}
% \fancyhf{} 
% \fancyhead[OL,EL]{\thepage}
% \fancyhead[OR]{\small\rightmark}
% \fancyhead[ER]{\small\leftmark}
% \renewcommand{\chaptermark}[1]{%
% \markboth{\thechapter-\ #1}{}}
% }
%%%%%%%%%%%%5
%\def\MATtextbaseline{1.5}
%\renewcommand{\baselinestretch}{\MATtextbaseline}
\linespread{1.9}
%%%%%%%%%%%%%%%%%%%%%%%%%%%%%
% دستوراتی برای اضافه کردن کلمه «فصل» در فهرست مطالب

\newlength\mylenprt
\newlength\mylenchp
\newlength\mylenapp

\renewcommand\cftpartpresnum{\partname~}
\renewcommand\cftchappresnum{\chaptername~}
\renewcommand\cftchapaftersnum{}

\settowidth\mylenprt{\cftpartfont\cftpartpresnum\cftpartaftersnum}
\settowidth\mylenchp{\cftchapfont\cftchappresnum\cftchapaftersnum}
\settowidth\mylenapp{\cftchapfont\appendixname~\cftchapaftersnum}
\addtolength\mylenprt{\cftpartnumwidth}
\addtolength\mylenchp{\cftchapnumwidth}
\addtolength\mylenapp{\cftchapnumwidth}

\setlength\cftpartnumwidth{\mylenprt}
\setlength\cftchapnumwidth{\mylenchp}	

\makeatletter
{\def\thebibliography#1{\chapter*{\refname\@mkboth
   { \uppercase{\refname}}{ \uppercase{\refname}}}\list
   {[\arabic{enumi}]}{\settowidth\labelwidth{[#1]}
   \rightmargin\labelwidth
   \advance\rightmargin\labelsep
   \advance\rightmargin\bibindent
   \itemindent -\bibindent
   \listparindent \itemindent
   \parsep \z@
   \usecounter{enumi}}
   \def\newblock{}
   \sloppy
   \sfcode`\.=1000\relax}}
\makeatother

\usepackage{perpage}
\MakePerPage{footnote}

\makeatletter
\newcommand{\nextverbatimspread}[1]{%
  \def\verbatim@font{%
    \linespread{#1}\normalfont\ttfamily% Updated definition
    \gdef\verbatim@font{\normalfont\ttfamily}}% Revert to old definition
}
\makeatother

%تنظیم فاصله متن تا خط زیرنویس
\setlength{\skip\footins}{1cm}


\usepackage{booktabs}

\renewcommand{\cfttoctitlefont}{\hfil \Huge \bfseries}
\renewcommand{\cftlottitlefont}{\hfil \Huge \bfseries}
\renewcommand{\cftloftitlefont}{\hfil \Huge \bfseries}



\graphicspath{{Figures/}}
\DeclareMathOperator*{\argmin}{arg\,min}
\DeclareMathOperator*{\argmax}{arg\,max}



\setcounter{tocdepth}{2}
\setcounter{secnumdepth}{5}

\newcommand{\cchapter}[1]{\chapter[#1]{\centering #1}}

\renewcommand\thefigure{\thechapter\lr{‑}\arabic{figure}}
\renewcommand\thetable{\thechapter\lr{‑}\arabic{table}}
\renewcommand\theequation{\thechapter\lr{‑}\arabic{equation}}
\renewcommand\thesection{\thechapter\lr{‑}\arabic{section}}
\renewcommand\thesubsection{\thesection\lr{‑}\arabic{subsection}}
\renewcommand\thesubsubsection{\thesubsection\lr{‑}\arabic{subsubsection}}

\renewcommand{\labelenumi}{\arabic{enumi}- }
\renewcommand{\labelenumii}{\alph{enumii}- }
\renewcommand{\labelenumiii}{\roman{enumiii}- }
\renewcommand{\labelitemi}{$\bullet$}
\renewcommand{\labelitemii}{$\circ$}


\makeatletter
\usepackage{patchcmd}
%nkhComment: This is not working anymore:
%\patchcommand{\@makecaption}{#1: #2}{#1: #2}{}{}
\makeatother
\usepackage{multirow}
\pagestyle{fancy}

%------------------------------
\lhead{}
% Length to control the \fancyheadoffset and the calculation of \headline
% simultaneously
\newlength\FHoffset
\setlength\FHoffset{1cm}

\addtolength\headwidth{2\FHoffset}

\fancyheadoffset{\FHoffset}

% these lengths will control the headrule trimming to the left and right 
\newlength\FHleft
\newlength\FHright

% here the trimmings are controlled by the user
\setlength\FHleft{1cm}
\setlength\FHright{0cm}

% The new definition of headrule that will take into acount the trimming(s)
\newbox\FHline
\setbox\FHline=\hbox{\hsize=\paperwidth%
  \hspace*{\FHleft}%
  \rule{\dimexpr\headwidth-\FHleft-\FHright\relax}{\headrulewidth}\hspace*{\FHright}%
}
\renewcommand\headrule{\vskip-0.6\baselineskip\copy\FHline}

\renewcommand{\chaptermark}[1]{\markboth{\thechapter\lr{‑}\ #1}{}}
%\renewcommand{\headrule}{\vbox to 0pt{\hbox to\headwidth{\dotfill}\vss}}
%------------------------------
%\setdigitfont{Arial}
\newcommand*\tageq{\refstepcounter{equation}\tag{\theequation}}

\usepackage{float}
\floatstyle{plaintop}
\restylefloat{table}

\usepackage{cite}

%\usepackage{mathptmx}
\usepackage[cal=cm, bb=ams, frak=esstix, scr=esstix]{mathalfa}

\usepackage{tabularx}


%تنظیم فاصله قبل و بعد فرمول‌ها 
\setlength{\abovedisplayskip}{6pt}
\setlength{\abovedisplayshortskip}{3pt}
\setlength{\belowdisplayshortskip}{3pt}
\setlength{\belowdisplayskip}{6pt}

\renewcommand{\contentsname}{\hspace*{\fill}فهرست مطالب\hspace*{\fill}} 
\renewcommand{\listfigurename}{\hspace*{\fill}فهرست اشکال\hspace*{\fill}} 
\renewcommand{\listtablename}{\hspace*{\fill}فهرست جداول\hspace*{\fill}}


\renewcommand{\cftdotsep}{1}
\renewcommand{\cftchapleader}{\bfseries\cftdotfill{\cftsecdotsep}}% dot leaders in bold
\renewcommand{\cftchapfont}{\bfseries}
\renewcommand{\cftsecfont}{\bfseries}

 \let\origaddvspace\addvspace
 \renewcommand{\addvspace}[1]{}
 
 %\let\origappendix\appendix % save the existing appendix command
\usepackage{floatpag}

\usepackage{placeins}